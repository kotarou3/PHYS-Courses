\documentclass[a4paper]{scrartcl}
\usepackage[cm]{fullpage}
\usepackage{amsmath, amssymb, esint}
\usepackage{siunitx}
\usepackage[greek, english]{babel}

\usepackage{sectsty}
\sectionfont{\large\selectfont}
\subsectionfont{\normalsize\selectfont}

\begin{document}

\title{PHYS3111: Basic X-Ray Diffraction Prework}
\author{ \\ \\}
\date{2017-03-27}
\maketitle

\section{Questions}
\subsection{If electrons in an X-ray tube are accelerated through a potential difference of \(V = \SI{20}{\kilo\volt}\), calculate the shortest possible wavelength appearing in the bremsstrahlung radiation}
The electron will acquire a kinetic energy of \(e V\), and then dump this all into a single photon of wavelength:
\[\lambda = \frac{h c}{e V} \approx \SI{0.62}{\angstrom}\]

\subsection{A cubic crystal reflects the K\textgreek{b} radiation of copper (\(\lambda = \SI{1.38e-10}{\metre}\)) at an angle of \(\theta = \SI{18}{\degree}\) in the first order (\(n = 1\)). Calculate the corresponding inter-planar spacing \(d\) for the crystal.}
The Bragg condition states that there will be maxima at angles satisfying \(2 d \sin \theta = n \lambda\). Solving for \(d\):
\[d = \frac{n \lambda}{2 \sin \theta} \approx \SI{2.23}{\angstrom}\]

\subsection{The crystal structure of NaCl is face-centred cubic (FCC). The unit cell is a cube with side length \(a\). The spacing \(d\) measured in this experiment is the separation between the upper face and that passing through the centre of the cube; thus \(d = \frac{a}{2}\).}
\subsubsection{Show that each unit cell of the crystal contains four `molecules' of NaCl.}
On average, each corner atom (Cl) is shared by 8 cells, each edge atom (Na) is shared by 4 cells and each face atom (Cl) is shared by 2 cells. The final atom in the centre (Na) is not shared.

Thus the amount of Na and Cl atoms per cell would be:
\begin{align*}
    N_{Na} &= 12 \cdot \frac{1}{4} + 1 = 4 \\
    N_{Cl} &= 8 \cdot \frac{1}{8} + 3 \cdot \frac{1}{2} = 4
\end{align*}
...which is 4 `molecules' of NaCl.

\subsubsection{Prove that the density of the NaCl crystal is given by \(\rho = \frac{M}{2 N_A d^3}\), where \(M\) is the molar mass and \(N_A\) is Avogadro's number.}
In a unit cell, we have 4 `molecules', so a mass of \(m = \frac{4 M}{N_A}\). A unit cell has volume \(V = a^3 = 8 d^3\). Thus our density is:
\[\rho = \frac{m}{V} = \frac{M}{2 N_A d^3}\]

\section{Experimental Plan}
\begin{itemize}
    \item Follow the method outlined in the operating instructions.
    \item Find the K lines and shortest wavelength radiation angles, and focus measurements around these points.
    \item Peaks are important for calculating the crystal spacing, while the spectrum is only important for verifying Kramer's law
\end{itemize}

\end{document}