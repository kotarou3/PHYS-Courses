\documentclass[a4paper]{scrartcl}
\usepackage[cm]{fullpage}
\usepackage{amsmath, amssymb, esint}
\usepackage{siunitx}

\usepackage{tikz, pgfplots, pgfplotstable}
\pgfplotsset{
    compat = 1.12,
    plot-scatter/.style = {
        only marks,
        mark size = 0.5
    }
}

\pgfplotstableread{data/water.tsv}\watertsv
\pgfplotstableread{data/acrylic.tsv}\acrylictsv
\pgfplotstableread{data/teflon.tsv}\teflontsv
\pgfplotstableread{data/nothing.tsv}\nothingtsv
\pgfplotstableread{data/tcnq+x.tsv}\tcnqpxtsv
\pgfplotstableread{data/tcnq-x.tsv}\tcnqnxtsv
\pgfplotstableread{data/tcnq+y.tsv}\tcnqpytsv
\pgfplotstableread{data/tcnq-y.tsv}\tcnqnytsv

\begin{document}

\title{PHYS3111: Nuclear Magnetic Resonance}
\author{ \\ \\ }
\date{2017-05-02}
\maketitle

\begin{abstract}
    We use nuclear (and electron) magnetic resonance to derive values for the magnetic moments of the hydrogen nucleus in water and acrylic, fluorine nucleus in Teflon, and electrons in TCNQ. Additionally, we exploit electron magnetic resonance to indirectly measure the horizontal strength of Earth's magnetic field at Kensington.
\end{abstract}

\section{Materials and Methods}
Please refer to the operating instructions of the experiment and the prework.

When taking measurements for the TCNQ sample, we tried to align the field generated by the Helmholtz coils to Earth's magnetic field, while everything else was taken in arbitrary alignments.

Since our equipment for measuring the resonant frequency artificially broadens the lines, there is not much point in measuring the line width. Instead, we will only focus on finding the maxima point, which should correspond to the resonant frequency.

Since our equipment returns the derivative of the absorption spectrum, we simply need to find the point it crosses zero. This can be done by taking a linear regression ``near'' the zero crossing, and then using its values for \(m\) and \(b\) to find the x-intercept and its corresponding error.

\section{Results}
\begin{table}
    \centering
    \begin{tabular}{c | c | c | c}
        Sample & \(B\) (\si{\milli\tesla}) & \(f\) (\si{\mega\hertz}) & Field Sweep (\si{\milli\tesla}) \\
        \hline
        Water & 327.000 & 14 & 0.5 \\
        Acrylic & 327.300 & 14 & 3.0 \\
        Teflon & 347.800 & 14 & 1.5 \\
        Nothing & 347.300 & 14 & 1.5 \\
        TCNQ (\(+x\)) & 1.760 & 50 & 0.2 \\
        TCNQ (\(-x\)) & 1.810 & 50 & 0.2 \\
        TCNQ (\(+y\)) & 1.790 & 50 & 0.2 \\
        TCNQ (\(-y\)) & 1.790 & 50 & 0.2
    \end{tabular}
    \caption{Field and frequency settings for each sample. \(\pm x\) means it was aligned to Earth's field, while \(\pm y\) was orthogonal.}
    \label{tab:settings}
\end{table}
\begin{table}
    \centering
    \begin{tabular}{c | c | c | c | c}
        Sample & Fit \(R^2\) & x-intercept (\si{\milli\tesla}) & \(B_0\) (\si{\milli\tesla}) & \(\mu\) (\si{\joule\per\tesla}) \\
        \hline
        Water & 0.9943 & \SI{0.038 \pm 0.011}{} & \SI{327.038 \pm 0.011}{} & \SI{1.41826 \pm 0.00005e-26}{} \\
        Acrylic & 0.9978 & \SI{-0.047 \pm 0.009}{} & \SI{327.253 \pm 0.009}{} & \SI{1.41733 \pm 0.00004e-26}{} \\
        Teflon & 0.9947 & \SI{0.026 \pm 0.005}{} & \SI{347.826 \pm 0.005}{} & \SI{1.33350 \pm 0.00002e-26}{} \\
        Nothing & 0.9427 & \SI{-0.037 \pm 0.029}{} & \SI{347.263 \pm 0.029}{} & \SI{1.33566 \pm 0.00011e-26}{} \\
        TCNQ (\(+x\)) & 0.9920 & \SI{-0.001 \pm 0.001}{} & \SI{1.759 \pm 0.001}{} & \SI{9.419 \pm 0.008e-24}{} \\
        TCNQ (\(-x\)) & 0.9891 & \SI{0.004 \pm 0.002}{} & \SI{1.814 \pm 0.002}{} & \SI{9.131 \pm 0.009e-24}{} \\
        TCNQ (\(+y\)) & 0.9926 & \SI{-0.002 \pm 0.001}{} & \SI{1.788 \pm 0.001}{} & \SI{9.263 \pm 0.007e-24}{} \\
        TCNQ (\(-y\)) & 0.9935 & \SI{-0.002 \pm 0.001}{} & \SI{1.788 \pm 0.001}{} & \SI{9.266 \pm 0.007e-24}{} \\
        TCNQ (corrected) & - & - & \SI{1.787 \pm 0.002}{} & \SI{9.272 \pm 0.008e-24}{}
    \end{tabular}
    \caption{Resonant frequency and magnetic moment results (ignoring the effects of Earth's field, except for TCNQ)}
    \label{tab:results}
\end{table}
\begin{figure}
    \centering
    \begin{tikzpicture}
        \begin{axis}[
            xlabel = \(\Delta B_0\) (\si{\milli\tesla}),
            ylabel = Resonance Derivative (a.u.)
        ]
            \addplot +[plot-scatter] table [
                x expr = \thisrowno{0} / 10
            ] {\watertsv};
        \end{axis}
    \end{tikzpicture}
    \begin{tikzpicture}
        \begin{axis}[
            xlabel = \(\Delta B_0\) (\si{\milli\tesla}),
            ylabel = Resonance Derivative (a.u.)
        ]
            \addplot +[plot-scatter] table [
                x expr = \thisrowno{0} / 10
            ] {\acrylictsv};
        \end{axis}
    \end{tikzpicture}
    \caption{Water and Acrylic data}
    \label{fig:water-acrylic}
\end{figure}
\begin{figure}
    \centering
    \begin{tikzpicture}
        \begin{axis}[
            xlabel = \(\Delta B_0\) (\si{\milli\tesla}),
            ylabel = Resonance Derivative (a.u.)
        ]
            \addplot +[plot-scatter] table [
                x expr = \thisrowno{0} / 10
            ] {\teflontsv};
        \end{axis}
    \end{tikzpicture}
    \begin{tikzpicture}
        \begin{axis}[
            xlabel = \(\Delta B_0\) (\si{\milli\tesla}),
            ylabel = Resonance Derivative (a.u.)
        ]
            \addplot +[plot-scatter] table [
                x expr = \thisrowno{0} / 10
            ] {\nothingtsv};
        \end{axis}
    \end{tikzpicture}
    \caption{Teflon and Nothing data}
    \label{fig:teflon-nothing}
\end{figure}
\begin{figure}
    \centering
    \begin{tikzpicture}
        \begin{axis}[
            xlabel = \(\Delta B_0\) (\si{\milli\tesla}),
            ylabel = Resonance Derivative (a.u.)
        ]
            \addplot +[plot-scatter] table [
                x expr = \thisrowno{0} / 10
            ] {\tcnqpxtsv};
        \end{axis}
    \end{tikzpicture}
    \begin{tikzpicture}
        \begin{axis}[
            xlabel = \(\Delta B_0\) (\si{\milli\tesla}),
            ylabel = Resonance Derivative (a.u.)
        ]
            \addplot +[plot-scatter] table [
                x expr = \thisrowno{0} / 10
            ] {\tcnqnxtsv};
        \end{axis}
    \end{tikzpicture}
    \caption{\(\pm x\) TCNQ data}
    \label{fig:tcnq-aligned}
\end{figure}
\begin{figure}
    \centering
    \begin{tikzpicture}
        \begin{axis}[
            xlabel = \(\Delta B_0\) (\si{\milli\tesla}),
            ylabel = Resonance Derivative (a.u.)
        ]
            \addplot +[plot-scatter] table [
                x expr = \thisrowno{0} / 10
            ] {\tcnqpytsv};
        \end{axis}
    \end{tikzpicture}
    \begin{tikzpicture}
        \begin{axis}[
            xlabel = \(\Delta B_0\) (\si{\milli\tesla}),
            ylabel = Resonance Derivative (a.u.)
        ]
            \addplot +[plot-scatter] table [
                x expr = \thisrowno{0} / 10
            ] {\tcnqnytsv};
        \end{axis}
    \end{tikzpicture}
    \caption{\(\pm y\) TCNQ data}
    \label{fig:tcnq-orthognal}
\end{figure}

The raw data collected with the settings shown in Table \ref{tab:settings} can be seen in Figures \ref{fig:water-acrylic} to \ref{fig:tcnq-orthognal}. Their corresponding results can be seen in Table \ref{tab:results}.

By aligning our TCNQ field, we were also able to observe the effects that Earth's magnetic field has on our measurements. No significant difference between the \(\pm y\) orientations were observed, with all the variation in the \(\pm x\) orientations, so we can conclude \(\pm x\) is correctly aligned to the Earth's field (at least in the plane). Half of the difference, \SI{28 \pm 1}{\micro\tesla}, would then correspond to the field strength.

\section{Discussion}
Our proton, flourine nucleus, and electron magnetic moments are quite close in magnitude values of \(\mu_p = \SI{1.411e-26}{\joule\per\tesla}\), \(\mu_F = \SI{1.3278e-26}{\joule\per\tesla}\) and \(\mu_e = \SI{-9.285e-24}{\joule\per\tesla}\) as given in the prework. Even though we were able to obtain the magnetic moments to uncanny precision, they are unlikely to be actually that precise (especially so since they don't actually match the quoted values) due to unaccounted-for systematic errors such Earth's magnetic field in the case of NMR, the calibration of our device, and chemical shift.

The ``nothing'' sample picked up a resonant frequency very close to that of Teflon, which suggests there is Teflon on the sample holder.

For Earth's magnetic field, the Australian Geomagnetic Reference Field model gives a value of \SI{24.756}{\micro\tesla} for the horizontal field at Kensington. Our value of \SI{28 \pm 1}{\micro\tesla} is close to this. Possible reasons for the discrepancy could include that the model is inaccurate (since it extrapolates data from the past, then interpolates them to get a higher resolution), or more likely, our experiment was under the influence of local variations to the field (such as the electromagnet used for NMR not being switched off during EMR).

\section{Conclusion}
\emph{(From abstract)} We used nuclear (and electron) magnetic resonance to derive values for the magnetic moments of the hydrogen nucleus in water and acrylic, fluorine nucleus in Teflon, and electrons in TCNQ. Additionally, we exploited electron magnetic resonance to indirectly measure the horizontal strength of Earth's magnetic field at Kensington.

The magnetic moment values can be found in Table \ref{tab:results}, and is close to the quoted values of \(\mu_p = \SI{1.411e-26}{\joule\per\tesla}\), \(\mu_F = \SI{1.3278e-26}{\joule\per\tesla}\) and \(\mu_e = \SI{-9.285e-24}{\joule\per\tesla}\). The discrepancy is most likely due to unaccounted-for systematic errors such Earth's magnetic field in the case of NMR, the calibration of our device, and chemical shift.

Meanwhile, we measured  \SI{28 \pm 1}{\micro\tesla} as Earth's horizontal magnetic field, but this differs from the Australian Geomagnetic Reference Field model gives a value of \SI{24.756}{\micro\tesla}. This is most likely due to local variations in the field that we measured, but the model does not account for.

\end{document}