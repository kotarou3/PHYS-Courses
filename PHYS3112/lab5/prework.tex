\documentclass[a4paper]{scrartcl}
\usepackage[cm]{fullpage}
\usepackage{amsmath, amssymb, esint}
\usepackage{siunitx}

\usepackage{sectsty}
\sectionfont{\large\selectfont}
\subsectionfont{\normalsize\selectfont}

\begin{document}

\title{PHYS3112: Compton Effect Prework}
\author{ \\ \\ }
\date{2017-05-07}
\maketitle

\section{Questions}
\subsection{Describe, concisely, the three main ways in which high energy photons interact with matter.}
\begin{itemize}
    \item Absorption or elastic scattering, where the photon is absorbed by an electron or the nucleus, and then possibly re-emitted (e.g., Photoelectric effect, X-ray crystallography)
    \item Inelastic scattering, where the wavelength does change (e.g., Compton effect)
    \item Pair production, where the photon decays in to a particle-antiparticle pair while simultaneously dumping excess energy or momentum in another particle
\end{itemize}

\subsection{Use the Compton formula to calculate the expected wavelength shift and energy change in this experiment for angles \SI{30}{\degree}, \SI{90}{\degree}, \SI{135}{\degree} and \SI{180}{\degree}.}
If we assume an incident energy of \(E_0 = \SI{662}{\kilo\electronvolt}\), then:
\[\Delta E = E_1 - E_0 = h c \left(\frac{1}{\lambda + \Delta \lambda} - \frac{1}{\lambda}\right)\]
\begin{center}
\begin{tabular}{c | c | c}
    Angle (\si{\degree}) & \(\Delta \lambda\) (\si{\pico\metre}) & \(\Delta E\) (\si{\kilo\electronvolt}) \\
    \hline
    30 & 0.325 & -97.9 \\
    90 & 2.436 & -373.6 \\
    135 & 4.142 & -455.9 \\
    180 & 4.853 & -477.7
\end{tabular}
\end{center}

\subsection{Suggest reasons for the broadening of the experimentally observed peaks.}
Primarily thermal Doppler broadening

\subsection{Briefly discuss the estimated resolution available with the combination of the Sodium Iodide scintillation detector and the Multi Channel Analyser. How does this compare to the expected peak separation?}
From the screenshot of the MCA, it appears to have a resolution of about \SI{0.217}{\kilo\electronvolt} per bin, with 4000 bins in total. This means we have a minimum and maximum possible separations of \SI{0.650}{\kilo\electronvolt} and \SI{866}{\kilo\electronvolt} respectively. This is enough to distinguish the separations we expect for angles between \SI{30}{\degree} and \SI{180}{\degree}.

\end{document}