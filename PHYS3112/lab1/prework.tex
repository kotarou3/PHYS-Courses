\documentclass[a4paper]{scrartcl}
\usepackage[cm]{fullpage}
\usepackage{amsmath, amssymb, esint}
\usepackage{siunitx}

\usepackage{sectsty}
\sectionfont{\large\selectfont}
\subsectionfont{\normalsize\selectfont}

\begin{document}

\title{PHYS3112: Electron Diffraction Prework}
\author{ \\ \\ }
\date{2017-03-06}
\maketitle

\section{Questions}
\subsection{Does the particle-wave duality apply to all particles? Calculate \(\lambda\) of a pitched baseball with mass \(m = \SI{0.15}{\kilo\gram}\) and velocity \(v = \SI{60}{\metre\per\second}\). What can you conclude from the result?}
The baseball has the following theoretical wavelength: \(\lambda = \frac{h}{m v} \approx \SI{7.36e-35}{\metre}\)

This is a tiny value, and to date has not been experimentally verified. As such, it is currently unknown whether the particle-wave duality applies to macroscopic objects. Realistically, this means we only ever observe the baseball's particle-like effects.

\subsection{Calculate \(\lambda\) of \SI{7.0}{\kilo\volt} electrons and compare it to the wavelength range of visible light. What is the advantage of an electron microscope compared to a microscope that uses visible light?}
A stationary electron accelerated through a voltage has energy:
\[E = m_e c^2 + e V = \sqrt{m_e^2 c^4 + p^2 c^2}\]
Solving for momentum:
\begin{align*}
    p &= \frac{\sqrt{e V (2 m_e c^2 + e V)}}{c} \\
    &\approx \sqrt{2 m_e e V} + \frac{(e V)^\frac{3}{2}}{2 \sqrt{2 m_e} c^2} + \mathcal{O}\left(V^\frac{5}{2}\right) \\
    &\approx \SI{4.5e-23}{\kilo\gram\metre\per\second}
\end{align*}
This is equivalent to a wavelength of:
\[\lambda = \frac{h}{\lambda} \approx \frac{h}{\sqrt{2 m_e e V}} \approx \SI{0.15}{\angstrom}\]

Meanwhile, visible light has wavelengths of about \SI{3900}{\angstrom} to \SI{7000}{\angstrom}.

As such, electrons provide much higher resolutions than visible light, since it is much less affected by diffraction.

Note the higher order terms only contributes \SI{0.015e-23}{\kilo\gram\metre\per\second} to the momentum, which is about \SI{0.34}{\percent}. This is much smaller than the expected accuracy of our experiment, so those terms can be safely ignored.

\subsection{Derive the relationship between the radii of the rings \(r_1\) and \(r_2\) and the accelerating potential \(V\). Explain how you can obtain the principal spacings of the graphite lattice, \(d_1\) and \(d_2\), from the plot of \(r_1\), \(r_2\) and \(V\).}
Assuming a spherical container of radius \(R\), the circles will subtend an angle of \(4 \theta = \sin^{-1} \frac{r}{R}\) to the opposite surface.

Meanwhile, the Bragg condition states that there will be maxima at angles satisfying \(2 d \sin \theta = n \lambda\). Substituting the above equations together gives:
\[2 d \sin \frac{\sin^{-1} \frac{r}{R}}{4} = \frac{n h}{\sqrt{2 m_e e V}}\]

If we let
\[x = 2 \sin \frac{\sin^{-1} \frac{r}{R}}{4}\]
\[y = \frac{n h}{\sqrt{2 m_e e V}}\]
then \(d\) will be the gradient of the linear regression between these two variables.

\section{Experimental Plan}
\begin{itemize}
    \item Follow the method outlined in the operating instructions.
    \item Vary G3 approximately linearly in a bisection pattern
\end{itemize}

\end{document}