\documentclass[a4paper]{scrartcl}
\usepackage[cm]{fullpage}
\usepackage{amsmath, amssymb, esint}
\usepackage{siunitx}

\usepackage{sectsty}
\sectionfont{\large\selectfont}
\subsectionfont{\normalsize\selectfont}

\begin{document}

\title{PHYS2114: RLC Circuits Prework}
\author{ \\ \\ }
\date{2016-09-12}
\maketitle

\section{Theoretical Questions}
\subsection{Show that the inductance \(L\) of an ideal close packed solenoid with an air core is given by \(L = \frac{\mu_0 \pi l}{4} \left(\frac{D}{d}\right)^2\)}
An ideal solenoid contains a B-field magnitude of:
\[B = \frac{\mu_0 N I}{l}\]

This is parallel with the solenoid, which gives a flux through the solenoid of:
\[\Phi_B = \frac{\mu_0 N I A}{l}\]

Combining this with the definition of inductance gives:
\[L = \frac{N \Phi_B}{I} = \frac{\mu_0 N^2 A}{l}\]
where \(N\) is the number of turns, and \(A\) the cross sectional area of the solenoid.

Given that the solenoid has a circular cross sectional area with diameter \(D\), it's area is:
\[A = \frac{\pi}{4} D^2\]

With wires of diameter \(d\), the number of turns becomes:
\[N = \frac{l}{d}\]

Thus we arrive at our result:
\[L = \frac{\mu_0 \pi l}{4} \left(\frac{D}{d}\right)^2\]

\subsection{Calculate the inductance of a solenoid of \(l = \SI{110}{\milli\metre}\), \(D = \SI{82.5}{\milli\metre}\) and \(d = \SI{0.5}{\milli\metre}\), assuming it is ideal}
\[L \approx \SI{2.96}{\milli\henry}\]

\subsection{For an series RLC circuit, assume \(R = \SI{12}{\ohm}\), \(L = \SI{2.56}{\milli\henry}\) and \(C = \SI{47}{\nano\farad}\). Calculate its resonant frequency \(f_1\), Q factor \(Q\) and bandwidth \(\Delta f\)}
\[f_1 \approx \SI{14.5}{\kilo\hertz}\]
\[Q \approx \SI{19.4}{}\]
\[\Delta f \approx \SI{746}{\hertz}\]

\section{Experimental Plan}
\begin{itemize}
    \item Note the impedance, capacitance and (if possible) the inductance of the oscilloscope (via model number and specification sheet?)
    \item Measure the actual parameters of the RLC circuit directly with a multimeter, if possible
    \item Save numerical oscilloscope data for later analysis
    \item Find the resonant frequency first before taking measurements, and then only points around that and extreme points (no point taking multiple measurements for ''flat'' zones)
    \item Try alternative resistances (including without the resistor!)
\end{itemize}

Otherwise, basically just follow what the operating instructions says to do.

\end{document}