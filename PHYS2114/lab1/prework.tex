\documentclass[a4paper]{scrartcl}
\usepackage[cm]{fullpage}
\usepackage{amsmath, amssymb, esint}
\usepackage{siunitx}

\usepackage{sectsty}
\sectionfont{\large\selectfont}
\subsectionfont{\normalsize\selectfont}

\begin{document}

\title{PHYS2114: Magnetic Hysteresis Prework}
\author{ \\ \\ }
\date{2016-08-01}
\maketitle

\section{Questions}
\subsection{What hysteresis loop shape would you require for a magnetic material to be used in a permanent magnet, or a transformer core? Give reasons for your choice.}
An ideal permanent magnet would be a straight line intersecting the \(B\) axis at the desired strength while a transformer core would be a straight line with no intersection requirements (since induction only depends on the time derivative of flux density). The gradient of this line would depend on the specific project requirements. A straight line would be optimal because it will not cause any waste power dissipation and is easy to model.

\subsection{Show that if \(V_{in} = V \cos(\omega t)\) for an RC integrator circuit, then \(V_{C} = \frac{V}{R C \omega} \sin(\omega t)\).}
\begin{align*}
    V_C &\approx \frac{1}{R C} \int_0^t V_{in} \:\mathrm{d}t \\
    &= \frac{V}{R C} \int_0^t \cos(\omega t) \:\mathrm{d}t \\
    &= \frac{V}{R C \omega} \sin(\omega t)
\end{align*}

\subsection{Show that \(R = \SI{22}{\kilo\ohm}\) and \(C = \SI{8}{\micro\farad}\) are suitable values for RC integration of a \SI{50}{\hertz} signal}
\begin{align*}
    \frac{1}{R C} &\approx \SI{5.68}{\radian\per\second} \\
    \omega &= 2 \pi \SI{50}{\hertz} \approx \SI{314}{\radian\per\second} \\
    \therefore \omega &\gg \frac{1}{R C}
\end{align*}

\subsection{Derive a scaling factor relating the area of the hysteresis loop and the energy dissipated per cycle per unit volume.}
The primary coil loops \(n_p\) times around a toroidal iron core with radius \(r\). It is powered by a constant-current source of \(I\), with it being measured by shunting the oscilloscope with a resistor of resistance \(R_p\) to provide a reading of \(V_X = I R_p\). This results in a H-field magnitude of:
\begin{align*}
    H &= \frac{n_p I}{2 \pi r} \\
    &= \frac{n_p}{2 \pi r R_p} V_X
\end{align*}

By Faraday's law of induction, the voltage at the secondary coil with \(n_s\) loops around the same core is:
\begin{align*}
    V_S &= n_s \frac{\mathrm{d}\Phi_B}{\mathrm{d}t} \\
    &= n_s W h \frac{\mathrm{d}B}{\mathrm{d}t}
\end{align*}

This then passes through the RC integrator, producing a final voltage reading of:
\begin{align*}
    V_Y &= \frac{1}{R C} \int_0^t V_S \:\mathrm{d}t \\
    &= \frac{n_s W h}{R C} B
\end{align*}
and rearranged for \(B\):
\[B = \frac{R C}{n_s W h} V_Y\]

Since we are interested in energy dissipated per cycle per unit volume:
\begin{align*}
    \frac{E}{V} &= \oint H \:\mathrm{d}B \\
    &= \frac{n_p}{n_s}\frac{R C}{2 \pi r W h R_p} \oint V_X \:\mathrm{d}V_Y
\end{align*}

Thus our scaling factor is:
\[\frac{n_p}{n_s}\frac{R C}{2 \pi r W h R_p} \approx \frac{n_p}{n_s} (\SI{4.67}{\kilo\joule\per\volt\squared\per\metre\cubed})\]

\subsection{Using the results from the previous section, how would you calculate the total energy dissipated by the iron sample over one hysteresis cycle?}
Since the iron sample is approximately a hollowed out cylinder with outer and inner radius of \(r + \frac{W}{2}\) and \(r + \frac{W}{2}\), its volume is:
\begin{align*}
    V &= \pi \left(\left(r + \frac{W}{2}\right)^2 - \left(r - \frac{W}{2}\right)^2\right) h \\
    &= 2 \pi r W h
\end{align*}

So total energy dissipated would be:
\[E = \frac{n_p}{n_s}\frac{R C}{R_p} \oint V_X \:\mathrm{d}V_Y\]
which is a scaling factor over the area by:
\[\frac{n_p}{n_s}\frac{R C}{R_p} = \frac{n_p}{n_s} (\SI{88}{\milli\joule\per\volt\squared})\]

\section{Experimental Plan}
\begin{itemize}
    \item Follow the method outlines in the student notes and operating instructions.
    \item Try more than just one frequency or other kinds of waveforms.
    \item Try more than one voltage.
    \item Integrate the actual data rather than weighing a printout...
    \item Try reading the voltage from the secondary coil directly rather than via the integrator.
    \item Consider the possibility of flux leakage from the iron sample.
\end{itemize}

\end{document}