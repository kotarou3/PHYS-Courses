\documentclass[a4paper]{scrartcl}
\usepackage[cm]{fullpage}
\usepackage{amsmath, amssymb, esint}
\usepackage{tikz}
\usepackage{siunitx}
\usepackage{cancel}

\usepackage{sectsty}
\sectionfont{\large\selectfont}
\subsectionfont{\normalsize\selectfont}

\begin{document}

\title{PHYS2113: Kater's Pendulum Prework}
\author{ \\ \\ }
\date{2016-05-10}
\maketitle

\section{Questions}
\subsection{Theoretical}
\subsubsection{Derive the equations for the period of the compound pendulum, \(T = 2 \pi \sqrt{\frac{l^2 + k_0^2}{g l}}\), as illustrated in the middle diagram in Figure 1.}
Let \(I_{CoM}\) be the moment of inertia around the centre of mass parallel to the pivot axis of the pendulum, and \(k_0\) be the radius of gyration around the centre of mass. Then by the parallel axis theorem, the moment of inertia around the pivot point is \(I = m l + I_{CoM} = m (l^2 + k_0^2)\).

The radius of oscillation of a pendulum is where if it is replaced by a simple pendulum of the same length, the period would remain the same. It can be shown that this radius is \(\frac{I}{m l}\).

Thus substituting in the radius of oscillation to the simple pendulum's period of \(T = 2 \pi \sqrt{\frac{l}{g}}\) produces \(T = 2 \pi \sqrt{\frac{l^2 + k_0^2}{g l}}\) for the compound pendulum.

\subsubsection{Suppose that \(T_1 = T_2\). What does equation (1) then imply for the determination of \(g\) when using Kater's pendulum?}
One does not need to know where the centre of mass of the pendulum to determine \(g\), since only \(l_1 + l_2\) would remain of the \(l\)s, and it is just the pivot-to-pivot distance.

\subsection{By setting \(L = l_1 + l_2\) show the relevant partial derivatives of \(g\) for the error analysis in Kater's pendulum.}
\begin{align*}
    g &= 8 \pi^2 \left(\frac{T_1^2 + T_2^2}{L} + \frac{T_1^2 - T_2^2}{2 l_1 - L}\right)^{-1} \\
    \therefore \frac{\partial g}{\partial T_1} &= -8 \pi ^2 \left(\frac{2 T_1}{2 l_1 - L} + \frac{2 T_1}{L}\right) \left(\frac{T_1^2 + T_2^2}{L} + \frac{T_1^2 - T_2^2}{2 l_1 - L}\right)^{-2} \\
    &= \frac{g^2}{8 \pi^2} \frac{4 T_1 l_1}{L (L - 2 l_1)}
\end{align*}
Continuing with similar substitution techniques:
\begin{align*}
    \frac{\partial g}{\partial T_2} &= -\frac{g^2}{8 \pi^2} \frac{4 T_2 (L - l_1)}{L (L - 2 l_1)} \\
    \frac{\partial g}{\partial L} &= \frac{g^2}{8 \pi^2} \left(\frac{T_1^2 + T_2^2}{L^2} - \frac{T_1^2 - T_2^2}{(L - 2 l_1)^2}\right) \\
    \frac{\partial g}{\partial l_1} &= \frac{g^2}{8 \pi^2} \left(2 \frac{T_1^2 - T_2^2}{(L - 2 l_1)^2}\right)
\end{align*}

\subsection{Experimental}
\subsubsection{How best to measure the periods of oscillation in order to minimise errors?}
Record the oscillations with a camera for about a minute each. Count the number of frames between the first and last peaks of oscillation, then divide by the number of peaks and the frame rate.

Image stabilisation software could be used if it isn't possible to place the camera on a stable surface.

Using a typical camera's 30 frames per second and a pendulum's period of approximately \SI{2}{\second} (that is, a total of 30 oscillations in the minute), this puts the error in period at around \SI{0.1}{\percent}.

This should be repeated a few times to determine a second error estimate.

\subsubsection{How best to measure the lengths in order to minimise errors?}
Unfortunately, only rulers and micrometres are available for use in the second year lab for measuring distances.

Unless we have a fancy laser rangefinder available for use, measuring pivot-to-pivot distance (\(L\)) will only be possible with a ruler. Since the pivots pivot on a knife's edge, the distance to measure would be from one knife edge to the other.

Assuming a pendulum length of approximately \SI{1}{\metre}, this would place the error at around \SI{0.1}{\percent} with a millimetre labelled ruler.

If the distance to the centre of mass (\(l_1\)) is close enough to one of the pivots, a micrometre could be used, but otherwise only a ruler can be used.

Finding the centre of mass would be done by placing the pendulum sideways on a knife's edge (without allowing it to roll) until it is parallel to the ground. Then the distance would be measured to one of the pivots' knife edge.

This should be repeated for multiple rotational orientations around the pendulum's axis to get an estimate on the error, since the pendulum is not perfectly symmetrical along its axis.

\section{Experimental Plan}
\begin{itemize}
    \item Follow as written in the operating instructions and student notes, as well as the above prework
    \item Use a (much more accurate) camera rather than a stopwatch for period measurement
    \item The stopwatch can also be used simultaneously for a second set of measurements, to be compared to the camera's measurements
    \item Define \(T_1\) to be the period when oscillating on the pivot closest to the mass (Pivot 1), and \(l_1\) the distance from that pivot to the centre of mass
    \item Take measurements again after adjusting the masses on the pendulum so that the difference between the two periods are as small as possible
    \begin{itemize}
        \item If \(T_2 < T_1\), or \(T_2 > T_1\), move the weight away from, or towards Pivot 2, respectively (``down'', or ``up'' the rod, when hanging from Pivot 2)
    \end{itemize}
\end{itemize}

\end{document}