\documentclass[a4paper]{scrartcl}
\usepackage[cm]{fullpage}
\usepackage{amsmath, amssymb, esint}
\usepackage{tikz}
\usepackage{siunitx}
\usepackage{cancel}

\usepackage{sectsty}
\sectionfont{\large\selectfont}
\subsectionfont{\normalsize\selectfont}

\begin{document}

\title{PHYS2113: Speed of Light Prework}
\author{ \\ \\ }
\date{2016-03-22}
\maketitle

\section{Questions}
\subsection{Prove that \(\Delta S = c \Delta t\), and state your assumptions.}
Suppose that light travels at a constant speed \(c\), then by basic mechanics:
\begin{align*}
    S &= c t \\
    S_k &= c t_k
\end{align*}
%
Combining this with equations (7):
\begin{align*}
    \cancel{S_k} + \Delta S &= \cancel{c t_k} + c \Delta t \\
    \Delta S &= c \Delta t
\end{align*}

\subsection{Sketch the plot of \(\Delta t\) vs \(\Delta S\).}
\begin{center}
    \begin{tikzpicture}
        \draw [->] (0, 0) -- (6, 0) node [midway, below] {\(\Delta t\) (Arb.)};
        \draw [->] (0, 0) -- (0, 4) node [midway, left] {\(\Delta S\) (Arb.)};
        \draw [-stealth, red] (0, 0) -- (5, 3) node [midway, left] {\(m = c\)};
    \end{tikzpicture}
\end{center}

\subsection{Derive the equation \(c_m = \frac{c_a l_m}{\Delta x + l_m}\) using the optical path length.}
First equate the two optical path lengths:
\[OPL = n_a (\cancel{x_1} + \Delta x) = n_m l_m + n_a (\cancel{x_1} - l_m)\]
\[\therefore \frac{n_a}{n_m} = \frac{l_m}{\Delta x + l_m}\]
%
Then consider refractive indices \(n_m\) and \(n_a\) in terms of the speed of light:
\[n_a = \frac{c}{c_a} \qquad n_m = \frac{c}{c_m}\]
\[\therefore \frac{n_a}{n_m} = \frac{c_m}{c_a}\]
%
Combining the two equations yields:
\[c_m = \frac{c_a l_m}{\Delta x + l_m}\]

\subsection{Prove that the optical path lengths in Measurements 1 and 2 are identical, despite the different physical distances travelled by the light.}
Since the same equation was derived using both the same optical path length condition and the non-optical path length conditions stated in Measurements 1 and 2, and since both are completely reversible, we can conclude that the conditions in Measurements 1 and 2 imply identical optical path lengths, despite the different physical distances.

\subsection{The intensity of the laser light is modulated periodically at a frequency of \SI{50}{\mega\hertz}. Why is the modulation necessary?}
The laser's actual emission frequency in the visible frequencies is much too high to be detected directly (i.e., without interferometry), so a low frequency modulation is used so that it can be detected by a cheap photodiode. Additionally \SI{50}{\mega\hertz} was chosen since its wavelength (\SI{6}{\metre}) is of the right order of magnitude to be used in the scales of mediums used in the experiment (~\SI{1}{\metre}).

\subsection{The output signals from the speed of light meter have their frequencies reduced by a factor of 1000 in order for them to be displayed on a simple oscilloscope such as that used in the experiment. Why not just modulate the intensity of the laser light itself at a frequency of \SI{50}{\kilo\hertz} then?}
\SI{50}{\kilo\hertz} has a wavelength of \SI{6}{\kilo\metre}, so detecting and correcting for the phase shift would be very difficult at the scales of our experiment.

\subsection{Suppose in addition to water and acrylic glass, you are supplied with a \SI{1.9}{\meter}-long tube of Titanium Dioxide for Part 2 of the experiment (ignore the fact that the optical bench scale only goes up to \SI{1.7}{\meter}). Now, the refractive index of Titanium Dioxide can be as high as \SI{2.6}{} in the visible frequencies. Would out equipment and the experimental set-up depicted in Figure 3 yield the correct speed of light in this material? From this consideration, discuss the limitations of this experiment and how you might get around them.}
The optical path length of the tube would be \(2.6 \times \SI{1.9}{\meter} \approx \SI{4.9}{\metre}\), producing a phase shift delay of \(2 \times 2 \pi \frac{(\SI{4.9}{\metre}) - (\SI{1.9}{\metre})}{\SI{6}{\metre}} = 2 \pi\), which would be indistinguishable from a phase shift of \(0\), and thus would produce a refractive index result of \(1\) from this experiment alone, which is obviously wrong.

From this, we can see that this experiment can only reliably measure refractive indices when the phase shift is \(< 2 \pi\). Since the phase shift is determined directly by the wavelength of the modulation and the optical path length difference (\(\varphi = 2 \times 2 \pi \frac{\Delta OPL}{\nu}\)), the problem can be resolved by decreasing the sample length (reducing \(\Delta OPL\)), or decreasing the modulation frequency (increasing \(\nu\)).

\section{Experimental Plan}
\begin{itemize}
    \item Follow the method outlines in the operating instructions.
    \item Measurement 1:
    \begin{itemize}
        \item Repeat with different \(\Delta S\) as many times as reasonable.
        \item Trying extrema then halving distances should make optimal use of time, since the expected speed of light result will be linear in \(\Delta S\) and \(\Delta t\).
        \item Speed of light result can be found by taking the linear regression of \(\Delta t\) vs \(\Delta S\) as the value, and \SI{95}{\percent} confidence intervals as the error.
    \end{itemize}
    \item Measurement 2:
    \begin{itemize}
        \item Repeat with different \(x_1\) as many times as reasonable.
        \item Trying extrema then halving distances should make optimal use of time, since \(\Delta x\) and refractive index should be constant regardless of \(x_1\).
        \item Relative refractive index result would be better found with \(n = \frac{\Delta x + l_m}{l_m}\) and assuming it close enough to the absolute refractive index rather than introducing our value and error in \(c_a\).
    \end{itemize}
    \item Errors will most likely be introduced in the measurement of \(\Delta S\), \(\Delta t\) and \(\Delta x\), assuming reliable equipment.
    \item A small amount of error will be introduced by the plastic container of the water, but should be negligible.
    \item There will be other sources of light such as from external reflections, but hopefully the SNR is high enough to see the intended measurement.
\end{itemize}

\end{document}